\begin{frame}
  \begin{block}{Tipos de utensílios domésticos}
    \begin{itemize}
      \item Eletrodomésticos podem ser divididos em três grupos:
      \begin{itemize}
        \item \textbf{Agendáveis:} Lava-louças, lava-roupas, ...
        \item \textbf{Reguláveis:} Ar-condicionado, aquecedor, ...
        \item \textbf{Essenciais:} TVs, fornos elétricos, lâmpadas, ...
      \end{itemize}
    \end{itemize}
  \end{block}
  \pause
  \begin{block}{Perfil de operação}
    \begin{itemize}
      \item Possuem características/restrições de período de utilização:
      \begin{itemize}
        \item \textbf{Agendáveis:} operação pode ser agendada/escalonada
        \item \textbf{Reguláveis:} tendem a ser utilizados em períodos fixos
        \item \textbf{Essenciais:} são utilizados em períodos específicos
        e demandam energia constante
      \end{itemize}
    \end{itemize}
  \end{block}
\end{frame}

\begin{frame}
  \begin{block}{Possíveis abordagens em \textit{Smart Grids}}
    \begin{itemize}
      \item Em uma rede inteligente, para cada grupo pode-se:
      \begin{itemize}
        \item \textbf{Agendáveis:} operadora pode agendar utilização para
        horários fora de pico, através de políticas preço (preço é um icentivo,
        mas conforto deve ser levado em consideração)
        \item \textbf{Reguláveis:} operadora pode otimizar parâmetros de
        utilização/consumo, sem contudo mudar o horário de funcionamento
        \item \textbf{Essenciais:} operadora não pode interferir na utilização,
        contudo pode prever a demanda diária e planejar a utilização da rede
        para garantir o balanço geração \textit{vs} consumo
      \end{itemize}
    \end{itemize}
  \end{block}
\end{frame}

%\begin{frame}
%  \begin{block}{}
%    \begin{itemize}
%      \item
%    \end{itemize}
%  \end{block}
%\end{frame}

%\begin{frame}
%  \begin{block}{}
%  \end{block}
%\end{frame}

%\begin{frame}
%  \begin{figure}[h]
%  	\begin{center}
%      \includegraphics [scale=0.3]{./Figures/Device-Estimates}
%     % \caption {Estimativa de dispositivos conectados à Internet.}
%  		%\label{fig:arq-imuno}
%  	\end{center}
%  \end{figure}
%\end{frame}

%\begin{frame}{Redes de Acesso}
%	\begin{figure}[!htb]
%		\centering
%		\subfloat[DSL]{
%			\includegraphics[height=3.5cm]{./Figures/DSLaccess}
%			\label{figdroopy}}
%		\quad %espaco separador
%		\subfloat[Cable]{
%			\includegraphics[height=3.5cm]{./Figures/CableAccess}
%			\label{figsnoop}}
%		%\caption{Subfiguras}
%		%\label{fig01}
%	\end{figure}
%\end{frame}

%\begin{frame}[fragile]
%\scriptsize
%\begin{verbatim}
%\end{verbatim}
%\end{frame}

%\begin{frame}{\textit{Socket Programming with TCP}}
%\scriptsize
%\lstinputlisting[language=Python, caption={TCP Server.}]{./code/upperServer/TCPserver.py}
%\end{frame}

